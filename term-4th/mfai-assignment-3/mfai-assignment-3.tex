\documentclass{article}
\usepackage[utf8]{vietnam}
\usepackage{amsmath}
\usepackage[nohead, nomarginpar, margin=1in, foot=.25in]{geometry}
\renewcommand{\baselinestretch}{1.15}
\begin{document}

\section*{Bài 3.} \large {Chéo hóa trực giao các ma trận sau:}
\[
    a) \; A =
    \begin{bmatrix}
        1 & 0 & 2 \\ 0 & -1 & -2 \\ 2 & -2 & 0
    \end{bmatrix}
    \quad
    b) \; B =
    \begin{bmatrix}
        2 & -1 & 0 \\ -1 & 2 & -1 \\ 0 & -1 & 2
    \end{bmatrix}
\]

\subsection*{Bài 3.a}
\[
    A =
    \begin{bmatrix}
        1 & 0 & 2 \\ 0 & -1 & -2 \\ 2 & -2 & 0
    \end{bmatrix}
\]
\( P_A(\lambda) = |A - \lambda I| = -\lambda^3 + 9\lambda.\)
\newline Khi đó \( P_{A}(\lambda) = 0 \Leftrightarrow \lambda \in \{ -3; 0; 3 \}. \)
\begin{itemize}
    \item Với \( \lambda = -3: \)
          \newline Giải hệ phương trình \( (A - \lambda I)x = 0 \).
          Ta có các vectơ riêng của A ứng với \( \lambda = -3\) có dạng
          \( x_1 = t.\begin{bmatrix} -1/2 \\ 1 \\ 1 \end{bmatrix}
          ,\forall t \in R, t \neq 0. \)
          \newline Chọn t = 1 và chuẩn hóa với \( \begin{Vmatrix} -1/2 \\ 1 \\ 1 \end{Vmatrix}
          = \dfrac{3}{2} \) , ta được
          \(v_1 = \begin{bmatrix} -1/3 \\ 2/3 \\ 2/3 \end{bmatrix}\)
    \item Với \( \lambda = 0: \)
          \newline Tương tự, ta có các vectơ riêng của A có dạng
          \( x_2 = t.\begin{bmatrix}
              -2 \\ -2 \\ 1
          \end{bmatrix} ,\forall t \in R, t \neq 0. \)
          \newline Chọn t = 1 và chuẩn hóa với \( \begin{Vmatrix} -2 \\ -2 \\ 1 \end{Vmatrix}
          = 3 \) , ta được
          \(v_2 = \begin{bmatrix} -2/3 \\ -2/3 \\ 1/3 \end{bmatrix}\).
    \item Với \( \lambda = 3: \)
          \newline Tương tự, ta có các vectơ riêng của A có dạng
          \( x_3 = t.\begin{bmatrix}
              1 \\ -1/2 \\ 1
          \end{bmatrix} ,\forall t \in R, t \neq 0. \)
          \newline Chọn t = 1 và chuẩn hóa với \( \begin{Vmatrix} 1 \\ -1/2 \\ 1 \end{Vmatrix}
          = \dfrac{3}{2} \) , ta được
          \(v_3 = \begin{bmatrix} 2/3 \\ -1/3 \\ 2/3 \end{bmatrix}\).
\end{itemize}
Ta thu được ma trận chéo hóa của A: \(
P = (v_1, v_2, v_3) = \begin{bmatrix}
    -1/3 & -2/3 & 2/3  \\
    2/3  & -2/3 & -1/3 \\
    2/3  & 1/3  & 2/3
\end{bmatrix} \)
\newline Thực hiện chéo hóa trực giao, ta được:
\[
    P^TAP = \begin{bmatrix}
        -3 & 0 & 0 \\ 0 & 0 & 0 \\ 0 & 0 & 3
    \end{bmatrix}
\]

\subsection*{Bài 3.b}
\[
    A =
    \begin{bmatrix}
        2 & -1 & 0 \\ -1 & 2 & -1 \\ 0 & -1 & 2
    \end{bmatrix}
\]
\( P_A(\lambda) = |A - \lambda I| = (2-\lambda)^3 - 2(2-\lambda). \)
\newline Khi đó \( P_{A}(\lambda) = 0 \Leftrightarrow \lambda \in \{ 2-\sqrt{2}, 2, 2+\sqrt{2} \}. \)
\begin{itemize}
    \item Với \( \lambda = 2-\sqrt{2}: \)
          \newline Giải hệ phương trình \( (A - \lambda I)x = 0. \)
          Ta có các vectơ riêng của A ứng với \( \lambda = 2-\sqrt2\) có dạng
          \( x_1 = t.\begin{bmatrix} 1 \\ \sqrt{2} \\ 1 \end{bmatrix}
          ,\forall t \in R, t \neq 0. \)
          \newline Chọn t = 1 và chuẩn hóa với \( \begin{Vmatrix} 1 \\ \sqrt2 \\ 1 \end{Vmatrix}
          = 2 \) , ta được
          \(v_1 = \begin{bmatrix} 1/2 \\ 1/ \sqrt2 \\ 1/2 \end{bmatrix}\)
    \item Với \( \lambda = 2: \)
          \newline Tương tự, ta có các vectơ riêng của A có dạng
          \( x_2 = t.\begin{bmatrix}
              -1 \\ 0 \\ 1
          \end{bmatrix} ,\forall t \in R, t \neq 0. \)
          \newline Chọn t = 1 và chuẩn hóa với \( \begin{Vmatrix} -1 \\ 0 \\ 1 \end{Vmatrix}
          = \sqrt2 \) , ta được
          \(v_2 = \begin{bmatrix} -1/ \sqrt2 \\ 0 \\ 1/ \sqrt2 \end{bmatrix}\)
    \item Với \( \lambda = 2+\sqrt2: \)
          \newline Tương tự, ta có các vectơ riêng của A có dạng
          \( x_3 = t.\begin{bmatrix}
              1 \\ - \sqrt2 \\ 1
          \end{bmatrix} ,\forall t \in R, t \neq 0. \)
          \newline Chọn t = 1 và chuẩn hóa với \( \begin{Vmatrix} 1 \\ - \sqrt2 \\ 1 \end{Vmatrix}
          = 2 \) , ta được
          \(v_3 = \begin{bmatrix} 1/2 \\ -1/ \sqrt2 \\ 1/2 \end{bmatrix}\)
\end{itemize}
Ta thu được ma trận chéo hóa của A: \(
P = (v_1, v_2, v_3) = \begin{bmatrix}
    1/2       & -1/ \sqrt2 & 1/2        \\
    1/ \sqrt2 & 0          & -1/ \sqrt2 \\
    1/2       & 1/ \sqrt2  & 1/2
\end{bmatrix} \)
\newline Thực hiện chéo hóa trực giao, ta được:
\[
    P^TAP = \begin{bmatrix}
        2 - \sqrt2 & 0 & 0 \\ 0 & 2 & 0 \\ 0 & 0 & 2 + \sqrt2
    \end{bmatrix}
\]

\section*{Bài 4.} \large {Phân tích kỳ dị các ma trận sau:}
\[
    a) \; A =
    \begin{bmatrix}
        1 & 0 & 1 \\ -1 & 0 & 1
    \end{bmatrix}
    \quad
    b) \; B =
    \begin{bmatrix}
        1 & 2 \\ 0 & 1 \\ -1 & 0
    \end{bmatrix}
\]

\subsection*{Bài 4.a}
\[
    A = 
    \begin{bmatrix}
        1 & 0 & 1 \\ -1 & 0 & 1
    \end{bmatrix}
\]
Ta có: \( M = AA^T = \begin{bmatrix}
    2 & 0 \\
    0 & 2
\end{bmatrix}\). Đa thức đặc trưng \( P_M(\lambda) = |M - \lambda I| = (2-\lambda)^2.\)
\newline \( P_M(\lambda) = 0 \Rightarrow \lambda = 2.\)
\newline Ta có các vectơ riêng của M ứng với \( \lambda = 2\) có dạng
\( x = a.\begin{bmatrix} 1 \\ 0 \end{bmatrix} + b.\begin{bmatrix} 0 \\ 1 \end{bmatrix}
,\forall a, b \in R, a^2 + b^2 \neq 0. \) Chọn (a, b) lần lượt = (1, 0) và (0, 1) và chuẩn hóa, ta được: 
\[ 
    u_1 = \begin{bmatrix} 1 \\ 0 \end{bmatrix}, \;
    u_2 = \begin{bmatrix} 0 \\ 1 \end{bmatrix}, \;
    U = (u_1, u_2) = \begin{bmatrix}
        1 & 0 \\ 0 & 1
    \end{bmatrix}
\]
Ta đồng thời có các giá trị kỳ dị \( \sigma_1 = \sigma_2 = \sqrt2 \) (bội đại số của \( \lambda = 2 \) là 2). 
\[
    \sum = \begin{bmatrix}
        \sqrt2 & 0 \\ 0 & \sqrt2
    \end{bmatrix}    
\]
Với ma trận V, ta có:
\begin{itemize} 
    \item \(
        v_1 = \dfrac{1}{ \sigma_1} A^Tu_1 = \dfrac{1}{ \sqrt2} 
        \begin{bmatrix}
            1 & -1 \\ 0 & 0 \\ 1 & 1
        \end{bmatrix}
        \begin{bmatrix}
            1 \\ 0
        \end{bmatrix}
        = \begin{bmatrix}
            1/ \sqrt2 \\ 0 \\ 1/ \sqrt2
        \end{bmatrix}
    \)
    \item \(
        v_2 = \dfrac{1}{ \sigma_2} A^Tu_2 = \dfrac{1}{ \sqrt2} 
        \begin{bmatrix}
            1 & -1 \\ 0 & 0 \\ 1 & 1
        \end{bmatrix}
        \begin{bmatrix}
            0 \\ 1
        \end{bmatrix}
        = \begin{bmatrix}
            -1/ \sqrt2 \\ 0 \\ 1/ \sqrt2
        \end{bmatrix}
    \)
\end{itemize}
V = \( (v_1, v_2)  = 
\begin{bmatrix}
    1/ \sqrt2 & -1/ \sqrt2 \\ 0 & 0 \\ 1/ \sqrt2 &  1/ \sqrt2
\end{bmatrix} \)
\newline Ta có phân tích kỳ dị:
\[ 
    A = U \sum V^T =
    \begin{bmatrix}
        1 & 0 \\ 0 & 1
    \end{bmatrix} 
    \begin{bmatrix}
        \sqrt2 & 0 \\ 0 & \sqrt2
    \end{bmatrix}    
    \begin{bmatrix}
        1/ \sqrt2 & 0 & 1/ \sqrt2 \\ -1/ \sqrt2 & 0 & 1/ \sqrt2
    \end{bmatrix} 
\]

\subsection*{Bài 4.b}
\[
    B = 
    \begin{bmatrix}
        1 & 2 \\ 0 & 1 \\ -1 & 0
    \end{bmatrix}
\]
Ta có: \( M = B^TB = \begin{bmatrix}
    2 & 2 \\
    2 & 5
\end{bmatrix}\). Đa thức đặc trưng \( P_M(\lambda) = |M - \lambda I| = (5- \lambda)(2 - \lambda) - 4. \) 
\newline \( P_M(\lambda) = 0 \Rightarrow \lambda \in \{ 6, 1 \}. \)
\begin{itemize}
    \item Ta có các vectơ riêng của M ứng với \( \lambda_1 = 6\) có dạng
\( x = t.\begin{bmatrix} 1/2 \\ 1 \end{bmatrix}
,\forall t \in R, t \neq 0. \) Chọn t = 1 và chuẩn hóa, ta được: 
\[ 
    v_1 = \begin{bmatrix}
        1/ \sqrt5 \\ 2 / \sqrt5
    \end{bmatrix}
\]
    \item Ta có các vectơ riêng của M ứng với \( \lambda_2 = 1\) có dạng
\( x = t.\begin{bmatrix} -2 \\ 1 \end{bmatrix}
,\forall t \in R, t \neq 0. \) Chọn t = 1 và chuẩn hóa, ta được: 
\[ 
    v_2 = \begin{bmatrix}
        -2/ \sqrt5 \\ 1 / \sqrt5
    \end{bmatrix}
\]
\end{itemize}
\[
    V = (v_1, v_2) = \begin{bmatrix}
        1/ \sqrt5 & -2/ \sqrt5 \\ 2 / \sqrt5 & 1 / \sqrt5
    \end{bmatrix}
\]
\newline Ta đồng thời có các giá trị kỳ dị \( \sigma_1 = \sqrt6 , \sigma_2 = 1 .\)  
\[
    \sum = \begin{bmatrix}
        \sqrt6 & 0 \\ 0 & 1
    \end{bmatrix}    
\]
Với ma trận U, ta có:
\begin{itemize} 
    \item \(
        u_1 = \dfrac{1}{ \sigma_1} Bv_1 = \dfrac{1}{ \sqrt6} 
        \begin{bmatrix}
            1 & 2 \\ 0 & 1 \\ -1 & 0
        \end{bmatrix}
        \begin{bmatrix}
            1/ \sqrt5 \\ 2 / \sqrt5
        \end{bmatrix}
        = \begin{bmatrix}
            5/ \sqrt30 \\ 2/ \sqrt30 \\ -1/ \sqrt30
        \end{bmatrix}
    \)
    \item \(
        u_2 = \dfrac{1}{ \sigma_2} BTv_2 =
        \begin{bmatrix}
            1 & 2 \\ 0 & 1 \\ -1 & 0
        \end{bmatrix}
        \begin{bmatrix}
            -2/ \sqrt5 \\ 1 / \sqrt5
        \end{bmatrix}
        = \begin{bmatrix}
            0 \\ 1/ \sqrt5 \\ 2/ \sqrt5
        \end{bmatrix}
    \)
\end{itemize}
U = \( (u_1, u_2)  = 
\begin{bmatrix}
    5/ \sqrt30 & 0 \\ 2/ \sqrt30 & 1 \ \sqrt5 \\ -1/ \sqrt30 & 2/ \sqrt5
\end{bmatrix} \)
\newline Ta có phân tích kỳ dị:
\[ 
    B = U \sum V^T =
    \begin{bmatrix}
        5/ \sqrt30 & 0 \\ 2/ \sqrt30 & 1 \ \sqrt5 \\ -1/ \sqrt30 & 2/ \sqrt5
    \end{bmatrix} 
    \begin{bmatrix}
        \sqrt6 & 0 \\ 0 & 1
    \end{bmatrix}    
    \begin{bmatrix}
        1/ \sqrt5 & 2/ \sqrt5 \\ -2 / \sqrt5 & 1 / \sqrt5
    \end{bmatrix} 
\]

\end{document}